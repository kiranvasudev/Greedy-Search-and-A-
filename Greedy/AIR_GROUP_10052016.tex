\documentclass[11pt]{article}
%Gummi|065|=)
\usepackage{graphicx}
\usepackage{hyperref}
\title{\textbf{Artificial Intelligence for Robotics\\ - Homework 6 -}}
\author{Kiran Vasudev, Patrick Nagel}
\date{Due date: 16.05.2016}
\begin{document}

\maketitle

\newpage
\tableofcontents


\newpage
\section{Search Strategies}
\subsection{Uniform Cost Search}
The uniform cost search is an \textbf{optimal} algorithm with any step-cost function. It stores and orders its nodes in a queue related to their path-costs. The node with the lowest path cost will be expanded. The algorithm then tests the node, which is selected, if it is the goal. Not to apply the goal test before the node with the lowest path-cost is selected avoids to choose a suboptimal solution. It makes sure that all nodes are generated and considered first. The uniform cost search also provides a test for the case that a better path to the goal is found. The  search focuses on the total costs of the nodes, which leads to the \textbf{possibility of infinite loops} if zero cost paths are given.\\
\\
The complexity of this algorithm does not depend on the depth of the search. Instead it is characterized by the path costs. In the worst case of the complexity is \textbf{$O(b^{1 + \frac{C*}{\epsilon}})$ (worst case)}. The C* represents the costs of an optimal solution and $\varepsilon$ the costs which are guaranteed at every node (can be also higher than that). In the case (that all costs are equal the complexity is \textbf{$O(b^{d + 1})$ (best case)}. The algorithm is then quite similar to breadth-first search just with the exception that the uniform cost search does not stop after finding the goal, but after checking all nodes (in case a lower cost solution exists). In this case the algorithm expands nodes unnecessarily since all costs are the same and the optimal path was alread found.


\subsection{Depth-First Search}
% Kiran

\subsection{Limited Depth-First Search}
The limited depth-first search is an extension of the depht-first search. It adds a depth limit to \textbf{avoid failure in infinite state spaces}. In the limit depth the children nodes are not longer generated, because it is the end of the search. \\
\\
The limited depth-first search contains a few scenarios, which can lead to different results and characteristics: Setting up a limit can result in not finding the goal at all. This case occurs if the goal node is beyond the limit. If the limit is selected higher than the actually depth, the search is also nonoptimal. In cases of having more information of the states, the limit can be chosen more efficient. Such a limit is called diameter.\\
\\
In general is the time complexity of the algorithm  $O(b^{l})$ and the space complexity $O(b \cdot l)$ (l = limit). In case that the limit is infite, there is no difference to the depth-first search any longer.  

\subsection{Iterative-deepening Search}
% Kiran

\subsection{Informed Search}
The expression informed search is the generic term for search algorithms which use problem-specific knowledge to find solutions more efficient. This types of searches are also called heurisitic searches. \\
\\
Informed searches use their knowledge to build an evaluation function. The function evaluates if it is worth to expand the node based on the costs, which have to be paid to reach the node and at the end the goal. The approach of choosing the most suitable way first is called best-first search. The evaluation function of lots of algorithms, which fit into the definition of a best-first search, contain a part which is called heuristic function. This function itself gives the estimated cost of the cheapest path from a node n to the goal node.

\subsection{Greedy Search}
% Kiran

\subsection{A*}
The A* search is the most famous search using the best-first approach. It was found 1968 by Peter Hart, Nils J. Nilsson and Bertram Raphael and contains as many other \textbf{best-first} search a heuristic function ($h(n)$). Together with a function, which gives the costs to reach a node ($g(n)$), it builds the evaluation function of the A* search: $f(n) = g(n) + h(n)$. In words the evaluation function $f(n)$ means the estimated cost of the chepeast solution through n.\\
\\
The algorithm finds always the best solution if such a  solution is given at all. Thus it is \textbf{optimal}. Furthermore the search is \textbf{complete} unless there are infinity nodes with $f(n) <= f(G)$. The complexity of the search is in the best case $O(d)$ and in worst case it is $O$ of the entire state space. The A* search can be compared to the uniformed-cost-search but instead of having only the g-function, it is g + h.

\subsection{Iterative A*}
% Kiran

\newpage
\section{What is a heuristic function?}
% Not decided yet.

\newpage
\section{Comparison of the two heuristics. (Task 3 (b))}
% Not decided yet.

\newpage
\section{Comment if the heuristics are consistent or inconsistent. (Task 3 (c))}
% Not decided yet.
\end{document}